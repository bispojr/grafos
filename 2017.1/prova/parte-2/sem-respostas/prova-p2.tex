\documentclass[12pt,a4paper,oneside]{article}

\usepackage[utf8]{inputenc}
\usepackage[portuguese]{babel}
\usepackage[T1]{fontenc}
\usepackage{amsmath}
\usepackage{amsfonts}
\usepackage{amssymb}
\usepackage{graphicx}

\usepackage{xcolor}
% Definindo novas cores
\definecolor{verde}{rgb}{0.25,0.5,0.35}
\definecolor{jpurple}{rgb}{0.5,0,0.35}
% Configurando layout para mostrar codigos Java
\usepackage{listings}
\lstset{
  language=Java,
  basicstyle=\ttfamily\small, 
  keywordstyle=\color{jpurple}\bfseries,
  stringstyle=\color{red},
  commentstyle=\color{verde},
  morecomment=[s][\color{blue}]{/**}{*/},
  extendedchars=true, 
  showspaces=false, 
  showstringspaces=false, 
  numbers=left,
  numberstyle=\tiny,
  breaklines=true, 
  backgroundcolor=\color{cyan!10}, 
  breakautoindent=true, 
  captionpos=b,
  xleftmargin=0pt,
  tabsize=4,
  escapeinside=||
}

\author{\\Universidade Federal de Goiás (UFG) - Regional Jataí\\Bacharelado em Ciência da Computação \\Teoria de Grafos \\Esdras Lins Bispo Jr.}

\title{\sc \huge Prova (Parte 2)}

\date{05 de setembro de 2017}

\begin{document}

\maketitle

{\bf ORIENTAÇÕES PARA A RESOLUÇÃO}

\footnotesize

\begin{itemize}
	\item A avaliação é individual, sem consulta;
	\item A pontuação máxima desta avaliação é 10,0 (dez) pontos, sendo uma das 06 (seis) componentes que formarão a média final da disciplina: quatro testes, uma prova e os exercícios de aquecimento;
	\item A média final ($MF$) será calculada assim como se segue
	\begin{eqnarray}
		MF & = & MIN(10, S) \nonumber \\
		S & = & (\sum_{i=1}^{4} 0,2.T_i ) + 0,2.P  + 0,1.EA \nonumber
	\end{eqnarray}
	em que 
	\begin{itemize}
		\item $S$ é o somatório da pontuação de todas as avaliações,
		\item $T_i$ é a pontuação obtida no teste $i$,
		\item $P$ é a pontuação obtida na prova, e
		\item $EA$ é a pontuação total dos exercícios de aquecimento.
	\end{itemize}
	\item O conteúdo exigido compreende os seguintes pontos apresentados no Plano de Ensino da disciplina: (3) Subgrafos, (4) Grafos Conexos e Componentes, (5) Cortes e Pontes, (6) Árvores, (7) Isomorfismo, (9) Planaridade, e (10) Outros tópicos.
\end{itemize}


\begin{center}
	\fbox{\large Nome: \hspace{10cm}}
	\fbox{\large Assinatura: \hspace{9cm}}
\end{center}

\newpage

\normalsize

\begin{enumerate}
	
	\section*{Terceiro Teste}
	
	\item (5,0 pt) {\bf [E 1.171]} Quantos componentes tem o grafo do cavalo 3-por-3? Quantos componentes tem o grafo do bispo $t$-por-$t$?
	
	\item (5,0 pt) Na linguagem C, complete as lacunas da função {\tt ehCircuito} conforme apresentada abaixo. Você deve substituir apenas as linhas 4, 8 e 11, pelos trechos de código 1, 2 e 3, respectivamente. O objetivo é que esta função retorne valor 1, se o grafo fornecido como parâmetro for um circuito. O valor de retorno será 0, caso não for.
	
	\begin{lstlisting}[language=C]
	int ehCircuito(Grafo *g){
	
		for(int v=0; v < g->n; v++){
			//TRECHO 1
		}
		
		if(ehConexo(g) == 1){
			//TRECHO 2
		}
		
		//TRECHO 3
	}\end{lstlisting}
	
	\newpage
	
	\section*{Quarto Teste}
	
	\item (5,0 pt) {\bf [E 1.196]} Seja $uv$ uma aresta de um grafo $G$. Mostre que $uv$ é uma ponte se e somente se $uv$ é o único caminho em G que tem extremos $u$ e $v$.
	
	\item (5,0 pt) Na linguagem C, complete as lacunas da função {\tt ehArvore} conforme apresentada abaixo. Você deve substituir apenas as linhas 4, 9 e 13, pelos trechos de código 1, 2 e 3, respectivamente. O objetivo é que esta função retorne valor 1, se o grafo fornecido como parâmetro for uma árvore. O valor de retorno será 0, caso não for.
	
	\begin{lstlisting}[language=C]
	int ehArvore(Grafo *g){
	
	if( g->n == 1)
		// TRECHO 1
	else{
		int vGrau1 = -1;
		for(int v=0; v<g->n; v++){
			if( grau(g,v) == 1){
			// TRECHO 2
			}
		}
	
		//TRECHO 3
		
		Grafo *h = gMenosV(g, vGrau1);
		return ehArvore(h);
	}\end{lstlisting}
	
	\end{enumerate}

	\newpage
	\section*{Anexos}
	\subsection*{grafos.h}
	
	\begin{lstlisting}[language=C]
	#ifndef GRAFO_H_INCLUDED
	#define GRAFO_H_INCLUDED
	
	#include <stdlib.h>
	
	typedef struct grafo {
	char *nome;
	int n;       // Numero de vertices
	int m;       // Numero de arestas
	int **adj;   // Ponteiro para a matriz de adjacencias
	} Grafo;
	
	int **gerarMatriz(int n);
	void inicializar(Grafo *g, char *nome, int n);
	void inserirAresta(Grafo *g, int v, int w);
	void removerAresta(Grafo *g, int v, int w);
	void exibir(Grafo *g);
	int grau(Grafo *g, int v);
	int grauMinimo(Grafo *g);
	int grauMaximo(Grafo *g);
	int ehRegular(Grafo *g);
	int eh3Regular(Grafo *g);
	Grafo *gMenosV(Grafo *g, int v);
	int ehCaminho(Grafo *g);
	int ehConexo(Grafo *g);
	int ehPonte(Grafo *g, int v, int w);
	
	#endif // GRAFO_H_INCLUDED\end{lstlisting}
	
	\newpage
	
	\subsection*{fabrica.h}
	
	\begin{lstlisting}[language=C]
	#ifndef FABRICA_H_INCLUDED
	#define FABRICA_H_INCLUDED
	
	#include <stdlib.h>
	#include "grafo.h"
	
	Grafo *k(int n);
	Grafo *caminho(int n);
	Grafo *circuito(int n);
	
	#endif // FABRICA_H_INCLUDED\end{lstlisting}
	
\end{document}