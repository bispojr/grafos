\documentclass[12pt,a4paper,oneside]{article}

\usepackage[utf8]{inputenc}
\usepackage[portuguese]{babel}
\usepackage[T1]{fontenc}
\usepackage{amsmath}
\usepackage{amsfonts}
\usepackage{amssymb}
\usepackage{graphicx}

\usepackage{xcolor}
% Definindo novas cores
\definecolor{verde}{rgb}{0.25,0.5,0.35}
\definecolor{jpurple}{rgb}{0.5,0,0.35}
% Configurando layout para mostrar codigos Java
\usepackage{listings}
\lstset{
  language=Java,
  basicstyle=\ttfamily\small, 
  keywordstyle=\color{jpurple}\bfseries,
  stringstyle=\color{red},
  commentstyle=\color{verde},
  morecomment=[s][\color{blue}]{/**}{*/},
  extendedchars=true, 
  showspaces=false, 
  showstringspaces=false, 
  numbers=left,
  numberstyle=\tiny,
  breaklines=true, 
  backgroundcolor=\color{cyan!10}, 
  breakautoindent=true, 
  captionpos=b,
  xleftmargin=0pt,
  tabsize=4,
  escapeinside=||
}

\author{\\Universidade Federal de Goiás (UFG) - Regional Jataí\\Bacharelado em Ciência da Computação \\Teoria dos Grafos \\Esdras Lins Bispo Jr.}

\title{\sc \huge Prova (Parte 2)}

\date{30 de agosto de 2016}

\begin{document}

\maketitle

{\bf ORIENTAÇÕES PARA A RESOLUÇÃO}

\footnotesize

\begin{itemize}
	\item A avaliação é individual, sem consulta;
	\item A pontuação máxima desta avaliação é 10,0 (dez) pontos, sendo uma das 05 (cinco) componentes que formarão a média final da disciplina: dois testes, duas provas e exercícios;
	\item A média final ($MF$) será calculada assim como se segue
	\begin{eqnarray}
		MF & = & MIN(10, S) \nonumber \\
		S & = & (\sum_{i=1}^{4} 0,2.T_i ) + 0,2.P  + EB \nonumber
	\end{eqnarray}
	em que 
	\begin{itemize}
		\item $S$ é o somatório da pontuação de todas as avaliações,
		\item $T_i$ é a pontuação obtida no teste $i$,
		\item $P$ é a pontuação obtida na prova, e
		\item $EB$ é a pontuação total dos exercícios-bônus.
	\end{itemize}
	\item O conteúdo exigido compreende os seguintes pontos apresentados no Plano de Ensino da disciplina: (5) Cortes e Pontes, (6) Árvores,  (7) Isomorfismo, (8) Coloração, (9) Planaridade e (10) Outros tópicos.
\end{itemize}

\begin{center}
	\fbox{\large Nome: \hspace{10cm}}
	\fbox{\large Assinatura: \hspace{9cm}}
\end{center}

\newpage

\normalsize

\begin{enumerate}

	\section*{Terceiro Teste}

	\item (5,0 pt) [E 1.106] Encontre o menor corte não trivial que puder no grafo do bispo $t$-por-$t$.

	\item (5,0 pt) [E 1.201] Suponha que todos os vértices de um grafo $G$ têm grau par. Mostre que G não tem pontes.
	
	\section*{Quarto Teste}

	\item (5,0 pt) [E 5.4] Suponha que $X$ e $Y$ são conjuntos estáveis maximais de um grafo. É verdade que $X$ e $Y$ são disjuntos (ou seja, que $X \cap Y = \emptyset$)?

	\item (5,0 pt) [DG 1.4 (Adaptação)] A função {\tt DIGRAPHcopy()} abaixo deveria receber um digrafo, criar uma cópia do digrafo, e devolver a cópia. Entretanto, há, ao menos, três erros nesta função. Identifique-os e corrija-os.
	
	\begin{lstlisting}
Digraph DIGRAPHcopy (Digraph g) {
	
	Digraph h;
	int i, j;
	
	h = DIGRAPHinit(g->A);
	
	for(i=0; i<g->V; i++){
		for(j=0; j<g->A; j++){
			g->adj[i][j] = h->adj[i][j];
		}	
	}
	
	return h;
}
\end{lstlisting}
	
	\end{enumerate}
\end{document}