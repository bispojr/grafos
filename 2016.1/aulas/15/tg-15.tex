\documentclass[xcolor=dvipsnames,table]{beamer}

\usepackage{latexsym}
\usepackage[utf8]{inputenc}
\usepackage[brazil]{babel}
\usepackage{amssymb}
\usepackage{amsmath}
\usepackage{stmaryrd}
\usepackage{fancybox}
\usepackage{datetime}
\usepackage[T1]{fontenc}
\usepackage{graphicx}
\usepackage{graphics}
\usepackage{url}
\usepackage{algorithmic}
\usepackage{algorithm}
\usepackage{acronym}
\usepackage{array}

\newtheorem{definicao}{Definio}
\newcommand{\tab}{\hspace*{2em}}

\mode<presentation>
{
  \definecolor{colortexto}{RGB}{0,0,0}
 
  \setbeamertemplate{background canvas}[vertical shading][ bottom=white!10,top=white!10]
  \setbeamercolor{normal text}{fg=colortexto} 

  \usetheme{Warsaw}
}

\title{Florestas e Árvores} 

\author{
  Esdras Lins Bispo Jr. \\ \url{bispojr@ufg.br}
  } 
 \institute{
  Teoria de Grafos \\Bacharelado em Ciência da Computação}
\date{\textbf{21 de junho de 2016} }

\logo{\includegraphics[width=1cm]{images/ufgJataiLogo.png}}

\begin{document}

	\begin{frame}
		\titlepage
	\end{frame}

	\AtBeginSection{
		\begin{frame}{Sumário}%[allowframebreaks]{Sumário}
    		\tableofcontents[currentsection]
    		%\tableofcontents[currentsection, hideothersubsections]
		\end{frame}
	}

	\begin{frame}{Plano de Aula}
		\tableofcontents
		%\tableofcontents[hideallsubsections]
	\end{frame}
	
	\section{Pensamento}
	\begin{frame}{Pensamento}
  		\begin{center}
    		\includegraphics[width=7cm]{images/pensamento.png}
  		\end{center}
	\end{frame}
	
	\begin{frame}{Pensamento}
		\begin{columns}
			\column{.4\textwidth}  		
		  		\begin{center}
		    		\includegraphics[height=.5\textheight]{images/desconhecido.jpg}
		  		\end{center}
			\column{.6\textwidth}  		
				\begin{block}{Frase}
					\begin{center}
						{\large Quando o machado entrou na floresta, as árvores disseram: \\- O cabo é dos nossos!!}
					\end{center}
				\end{block}		  		
		  		\begin{block}{Quem?}
		  			\begin{center}
						{\bf Provérbio Turco}
					\end{center}
				\end{block}
		\end{columns}
	\end{frame}
    
    \section{Revisão}
    	\subsection{Pontes}
	\begin{frame}{Pontes}
		\begin{block}{Definição}
			Uma {\bf ponte} ($bridge$) em um grafo $G$ é qualquer aresta $e$ tal que 
			\begin{center}
				$c(G - e) > c(G)$,  
			\end{center}
			ou seja, $G - e$ tem mais componentes que $G$.				
		\end{block}
		\begin{block}{Outros nomes}
			\begin{itemize}
				\item {\bf istmo} ({\it isthmus}), ou 
			 	\item {\bf aresta de corte} ({\it cut edge}).
			 \end{itemize}
		\end{block}
	\end{frame}
	
	\begin{frame}{Pontes}
		\begin{block}{Corolário}
			Uma aresta $a$ é ponte se e somente se o conjunto $\{ a \}$ é um corte do um grafo.
		\end{block}
		\begin{block}{Pontes $\times$ Circuitos}
			Em qualquer grafo, toda aresta é uma ponte ou pertence a um circuito, mas não ambos (E. 1.199).
		\end{block}
	\end{frame}
	
	\subsection{Trilhas}
	\begin{frame}{Trilhas}
		\begin{block}{Passeio}
	Um {\bf passeio} ({\it walk}) em um grafo é qualquer sequência finita $(v_0, v_1, v_2, \ldots, v_{k-1}, v_k)$ de vértices tal que $v_i$ é adjacente a $v_{i-1}$ para todo $i$ entre 1 e $k$.
		\end{block}
		\begin{block}{Detalhe}
			Os vértices do passeio podem não ser distintos dois a dois.
		\end{block}
		\begin{block}{Trilha}
			Uma {\bf trilha} ({\it trail}) é um passeio sem arestas repetidas.
		\end{block}
	\end{frame}
	
	\begin{frame}{Trilhas}
		\begin{block}{Passeio ou trilha fechados}
			\begin{itemize}
				\item Um passeio é fechado se $v_0 = v_k$;
				\item Uma trilha é fechada se $v_0 = v_k$;
			\end{itemize}			
		\end{block}
		\begin{block}{Expressões comuns}
			\begin{itemize}
				\item $v_0$ é a {\bf origem} do passeio;
				\item $v_k$ é o {\bf término} do passeio;
				\item o passeio {\bf vai de} $v_0$ a $v_k$;
				\item o passeio {\bf liga} $v_0$ a $v_k$;
			\end{itemize}
		\end{block}
	\end{frame}
	
	\begin{frame}{Trilhas}
		\begin{block}{Passeio simples}
			Um passeio é {\bf simples} se os seus vértices são distintos dois a dois.
		\end{block}
		\begin{block}{Ciclo}
			Um {\bf ciclo} é uma trilha fechada.
		\end{block}
		\begin{block}{Ciclo Euleriano}
			Um ciclo é {\bf euleriano} se e somente se passa por todas as arestas do grafo.
		\end{block}
	\end{frame}
	
	\section{Florestas e Árvores}
	\begin{frame}{Florestas e Árvores}
		\begin{block}{Floresta}
			\begin{itemize}
				\item Uma {\bf floresta} ($forest$) é um grafo sem circuitos. \pause
				\item Também chamado de grafo acíclico. \pause
				\item Um grafo é uma floresta se cada uma de suas arestas é uma ponte. 
			\end{itemize}
		\end{block} \pause
		\begin{block}{Árvore}
			Uma {\bf árvore} ($tree$) é uma floresta conexa. 
		\end{block} \pause
		\begin{block}{Corolário 1}
			Cada componente de uma floresta é uma árvore.
		\end{block}
	\end{frame}
	
	\begin{frame}{Florestas e Árvores}
		\begin{block}{Folha}
			Uma {\bf folha} ($leaf$) de uma floresta é qualquer vértice da floresta que tenha grau 1.
		\end{block} \pause
		\begin{block}{Corolário 2}
			Um grafo $G$ é uma floresta se e somente se $m(G) = n(G) - c(G)$.
		\end{block}
	\end{frame}
	
	\section{Planaridade}
	\begin{frame}{Grafos Planares}
		\begin{block}{Definição (informal)}
			Um grafo é {\bf planar} se pode ser desenhado no plano sem que as linhas que representam arestas se cruzem.
		\end{block} \pause
		\begin{block}{Exercícios}
			\begin{itemize}
				\item Todo caminho é planar? Todo circuito é planar? \pause
				\item Toda grade é planar? \pause
				\item Todo $K_4$ é planar? Todo $K_5$ é planar? \pause
				\item Todo $K_{2,3}$ é planar? Todo $K_{3,3}$ é planar?
			\end{itemize}
		\end{block}
	\end{frame}
	
	\begin{frame}
		\titlepage
	\end{frame}
	
\end{document}